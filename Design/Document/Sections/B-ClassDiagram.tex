\section{Classes}

\subsection{Class diagram}

\subsection{Fassade}

\subsection{Parser}

\subsubsection*{Parser}
Parser is the 'main'-Class of the Parser. It creates and administrates a threadpool of ParsingThreads,
contains the entryBuffer for finished DataEntries, the stringBuffer for strings, whcih were extracted
from the logfile, a method to add them to the data warehouse
and allows the user to select a new logfile and configuration file.

\subsubsection*{LogFile}
The gateway between parser and logfile - it contains the path of the logfile and an integer, 
which saves how many lines have been read from this file. It can read single lines from the logfile and return them to 
the Parser.

\subsubsection*{DataEntry}
The dataEntry which will be written in the warehouse. It contains hour, day, month, year of the request, rows which were read 
from the logfile, the elapsed and busy time on the server, the ip, from which the request came, the type of request, the database
and server which handle the request and an object-array of additional information, which depends on the actual database entry.

\subsubsection*{Java Files}
The Parser uses java.io.File and a ThreadPool from java.util.

\subsubsection*{ParseTools}
The ParseTools are a set of static tools to parse the logline.

\subsubsection*{NormalFormTool}
The NormalFormTool is used first. It converts the logline in a standardized form, which can be used by the other tools.

\subsubsection*{VerificationTool}
The VerificationTool checks the logline and looks if it is a correct logline - if the logline got a mistake, it will 
be deleted and the VerificationTool sends an error.



\subsubsection*{ParsingThread}
A parsing thread is one of the threads which are created from Parser. It gets a line from the logfile and uses the ParseTools
to create a dataEntry, which represents the line from the log. 
