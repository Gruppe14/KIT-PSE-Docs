\section{Classes}


%=||======================================================================>
\subsection{Server tier class diagram}
This diagram shows the webserver tier of WHAT. It controls the webserver and it can only access the facade of the tier below.
\begin{center}
\includegraphics[width=1\linewidth]{Pictures/ServerTierDia.png}
%\includegraphics[width=1\linewidth]{Pictures/ServerTierDiaNormal.png} 
\end{center}   


\subsubsection*{WebpageControllers}
All valid HTTP requests are mapped to a method from WebpageControllers.
WebpageControllers either uses other helper classes to handle requests itself  
or makes calls to the facade of the application tier. 
The last step done by WebpageControllers is to create a valid HTTP response.
                                                                            

\subsubsection*{HTML views}
The HTML views contain the main template for, and all other HTML content of the webpage. %unklar
They do not have to be static, but can also dynamically integrate some content, e.g. localized strings.

\subsubsection*{Static files}
Static files contains all files of the webpage, which are not changed during runtime. 
This includes, but is not limited, to images of the webpage, javascript files and stylesheets.

\subsubsection*{Localize}
Localize is a static helper class handling all the language related things. 
It has a method to get localized Strings using the language files, 
to change the language of the webpage und methods determining all available languages, 
%dieser Satz, passt der syntaktisch? Sollte es methods sein?
so they can be integrated in the webpage once they are present.

\subsubsection*{ChartIndex}
ChartIndex implements the singleton pattern. On instantiation it scans which chart types are available 
and saves them internally. This is used by the index page to dynamically display all available chart types,
even if one is deleted or added.
%Do we have to specify how one might add them? I mean, not here, but somewhere.

%=||======================================================================>
\subsection{Application \& Data access tier class diagram}
On the next pages the class diagram of the application and data acces tier are displayed. 
\newpage
\begin{center}
\includegraphics[width=0.9\linewidth]{Pictures/AppTierDia1.png}
%\includegraphics[width=0.9\linewidth]{Pictures/AppTierDia1Normal.png} 
\end{center}  

\begin{center}
\includegraphics[width=0.9\linewidth]{Pictures/AppTierDia2.png}
%\includegraphics[width=0.9\linewidth]{Pictures/AppTierDia2Normal.png} 
\end{center}  

%=||======================================================================>
\subsection{Facade + Configurations}
The facade of the application tier and configuration files in this design show a
optional feature. According to the database on which is operated - if there are more
than SkyServer - just a configuration file for this specific one, has to be created and
given to the program. This may give the application to power to manage all functions dynamically
according to the database. Facade and ConfigDB handle this configurations.


\begin{center}
\includegraphics{Pictures/Parts/FacadeConfi.png}
\end{center}   

\subsubsection*{Facade}
The Facade is, as you might have suggested, the facade of the application tier. 
This means the only way to communicate with this tier is via this class.
On of its task is to get the right ConfigDB for any call. With this he forwards
them to the corresponding mediators or workers. So new log files are passed to
the ParserMediator and chart requests to the ChartMediator. Some other optional
calls, like the history of charts or the configuration of a new Warehouse are just
adumbrated.


\subsubsection*{ConfigDB}
The ConfigDB class provides a static factory for creating itself out of a configuration
file, which may use caching. Objects of ConfigDB store the information about the database
they describe and while be used by all the dynamic methods, which are hooked on specific
databases. 

One thing stored in a ConfigDB is a String indicating the form of a line in the log files.
Other attributes are booleans for specific measures or dimensions, as well as trees of
Strings for dimensions like Databas > Server or the Type.


\begin{center}
\includegraphics{Pictures/Parts/Strings.png}
\end{center}  
\subsubsection*{Server, Database, Type, Country, City}
All this classes are Stringwrappers and also build as trees. They stand exemplary for
possible things stored in a ConfigDB. The implementation of them may be way more dynamic.
Whereas the content of Server \& Database and Type are depended on the configuration,
the Country \& City tree is mostly hooked to the GeoIP library.


%=||======================================================================>
\subsection{Chart request operator}
The task of the following classes is to handle a chart request. For their design the visitor pattern
is used. 

\begin{center}
\includegraphics[width=1\linewidth]{Pictures/Parts/MediVisi.png}
\end{center}  

\subsubsection*{ChartMediator}
The ChartMediator is the mediator of the whole chart request process. For a incoming request it will create a new
ChartHost firstly. Then he triggers the visits of the three visitors. 


\subsubsection*{ChartVisitor}
ChartVisitors work on a ChartHost. What they do depends on which specific one the visit and of course their
one type. One visitor, the ChartDataRequester, is part of the data access tier.

\subsubsection*{DataPreparer}
The DataPreparer is the second visitor. He prepares the raw data stored in a ChartHost, so that the JSON-Writter
can do its work easily.

\subsubsection*{JSON-Writter}
The JSON-Writter is the last visitor. According to the data stored in the visited ChartHost he creates a .json file
which contains all information needed by D3 to display the charts. Therefor he is supported by the JSON-Library.


\begin{center}
\includegraphics{Pictures/Parts/Petra.png}
\end{center}  
\subsubsection*{ChartVisitor}
\subsubsection*{ChartHost/Petra}
\subsubsection*{OneDimeChart}
\subsubsection*{TwoeDimeChart}
\subsubsection*{ThreeDimeChart} 


\subsection{Chart parameters}
\begin{center}
\includegraphics[width=1\linewidth]{Pictures/Parts/ChartPara.png}
\end{center}  
\subsubsection*{ChartParam} 


%=||======================================================================>
\subsection{Parser}

\begin{center}
\includegraphics{Pictures/Parts/ParsMedi.png}
\end{center}  

\subsubsection*{ParserMediator}
ParserMediator is the 'main'-Class of the Parser. It creates and administrates a threadpool,
 which contains several tasks, 
%contains used twice very shortly (I was momentarily confused).Also, the sentence is huge. I see no immediate replacement
contains the entryBuffer for finished DataEntries, the stringBuffer for strings,
which were extracted from the logfile and saves which log file and configuration file is used.

\subsubsection*{LogFile}
The gateway between parser and logfile - it contains the path of the logfile and an integer 
which saves how many lines have been read from this file. It can read single lines from the
logfile and return them to the Parser.

\begin{center}
\includegraphics{Pictures/Parts/DataEntry.png}
\end{center}  

\subsubsection*{DataEntry}
The dataEntry which will be written in the warehouse.
It contains 
-hour, day, month, year of the request,
-rows which were read  from the logfile%from the logfile or rows accessed in the original db? %+1, it is not clear
-the elapsed and busy time on the server
-the ip from which the request came
-the type of request
-the database and server which handled the request. %I Thought I'd break this up and make a list
It may contain additional information depending on the actual logfile, as specified in the configuration file. 
%is this correct?

%\newline\newline doesnt work in empty lines
% \subsubsection*{Java Files}
% The Parser uses java.io.File and a ThreadPool from java.util.

\begin{center}
\includegraphics[width=1\linewidth]{Pictures/Parts/TaskTool.png}
\end{center}  

\subsubsection*{ParsingTask}
A ParsingTask is one of the tasks which are created from Parser. It gets a line from the logfile and uses, SplittingTool, VerifyTool
and LocationTool to create a dataEntry which contains the same information. 


\subsubsection*{SplittingTool}
The splitting tool splits the dataLine from its parsingTask and enters the splitted parts into the dataEntry.

\subsubsection*{VerifyTool}
The VerifyTool checks if the dataEntry is correct - if it has a mistake (e.g. Month = 13 or Elapsed Time = -1), it will 
be deleted and the VerifyTool sends an detailed error, which can be displayed elsewhere.

\subsubsection*{LocationTool}
The LocationTool takes the IP from the dataEntry and uses GeoIP-libraries[geo] to determine the city and country of the request.
Those will be added to the dataEntry.


\subsubsection*{LoadingTask}

A loading task is another possible task for the threadPool. It takes a finished dataEntry from the buffer and sends it to the warehouse.




%=||======================================================================>
\subsection{Data access tier}

\begin{center}
\includegraphics{Pictures/Parts/Data.png}
\end{center} 

\subsubsection*{ChartDataRequester}

The ChartDataRequester is the first of the three visitors of a ChartHost. Its task is to 
create the queries to gain the data needed for the charts. It takes the informationen about what he
has to request from the hosted ChartHost.



\subsubsection*{OracleWHAdapter}

Every query, whether loading, extracting or anythin else, runned against the Oracle server is made by this,
and only by this, class. So it represents the interface out of the application to the data.



\subsubsection*{JavaToOracle}

Any library used for the connection and communication with the Oracle server, 
where the Warehouse is stored. It is only used by the OracleWHAdapter.



\subsubsection*{WareHouseConfiger}

This class will only be needed if a very optional function will be implemented. It's task is to create
a WareHouse for one certain new database just with the information of its configuration file information.



geo
 link to the maxmind geoip library in the library section
