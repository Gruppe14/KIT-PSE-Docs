\section{Classes}

\subsection{Class diagram}

\subsection{webpage} % don't know if this should be here, move as desired, also correct if it is not sufficient

\subsubsection*{WebpageControllers}
All valid HTTP requests are mapped to a method from WebpageControllers.
WebpageControllers uses other helper classes to handle requests itself 
or makes calls to the facade of the application tier. 
The last step done by WebpageControllers is to create a valid HTTP response.

\subsubsection*{HTML views}
the HTML views contains the main template for, and all other HTML content of the webpage. 
They do not have to be static, but can also dynamically integrate some content, e.g. localized strings.

\subsubsection*{static files}
static files contains all files of the webpage, which are not changed in runtime. 
This includes, but is not limited to images of the webpage, javascript files and stylesheets.

\subsubsection*{Localize}
Localize is a static helper class handling all the language related things. 
It has a method to get localized Strings using the language files, 
to change the language of the webpage und methods determining all available languages, 
so they can be integrated in the webpage once they are present.

\subsubsection*{ChartIndex}
ChartIndex implements the singleton pattern. On instanciation it scans which chart types are available 
and saves them internally. This is used by the index page to dynamically display all available chart types,
even if one is deleted or added.

\subsection{Fassade + Configurations}
%picture
\subsubsection{Fassade}


\subsection{Chart request operator}
Visitor pattern blabla
classes
%picture
\subsubsection{Fassade}

\subsection{Chart parameters}
The chart parameters describe 

\subsection{}


\subsection{Parser}

HIER LOGFILE & PARSERMEDIATOR HIN!
CAPSLOCK REGELT, UND SO!!11einself

\subsubsection*{ParserMediator}
ParserMediator is the 'main'-Class of the Parser. It creates and administrates a threadpool, which contains several tasks, 
contains the entryBuffer for finished DataEntries, the stringBuffer for strings, which were extracted
from the logfile and saves which log file and configuration file is used.

\subsubsection*{LogFile}
The gateway between parser and logfile - it contains the path of the logfile and an integer 
which saves how many lines have been read from this file. It can read single lines from the logfile and return them to 
the Parser.

DATAENTRY

\subsubsection*{DataEntry}
The dataEntry which will be written in the warehouse. It contains hour, day, month, year of the request, rows which were read 
from the logfile%from the logfile or rows accessed in the original db?
, the elapsed and busy time on the server, the ip, from which the request came, the type of request, the database
and server which handled the request and maybe additional information depending on the actual logfile.

\newline\newline
\subsubsection*{Java Files}
The Parser uses java.io.File and a ThreadPool from java.util.

TOOLS + PARSING TASK!

\subsubsection*{SplittingTool}
The splitting tool splits the dataLine from it's parsingTask and enters the splitted parts into the dataEntry.

\subsubsection*{VerifyTool}
The VerifyTool checks, if the dataEntry is correct - if it has a mistake (e.g. Month 13 or Elapsed Time -1), it will 
be deleted and the VerifyTool sends an detailed error, which can be displayed elsewhere.

\subsubsection*{LocationTool}
The LocationTool takes the IP from the dataEntry and uses GeoIP-libraries to determine the city and country of the request.
Those will be added to the dataEntry.

\subsubsection*{ParsingTask}
A ParsingTask is one of the tasks which are created from Parser. It gets a line from the logfile and uses, SplittingTool, VerifyTool
and LocationTool to create a dataEntry which contains the same information. 

LOADING TASK!

\subsubsection*{LoadingTask}
A loading task is another possible task for the threadPool. It takes a finished dataEntry from the buffer and sends it to the warehouse.
