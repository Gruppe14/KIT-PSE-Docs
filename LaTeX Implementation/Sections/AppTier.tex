\section{Changes in and comments to the Application tier}



%\textit{Getters & Setters} : Added many getters and setters.


\subsection{Working with a config}
At the beginning of the implementation phase, we decided to make our
application run very dynamically. To ensure this, we wanted to make nearly
every work be based on a given configuration file.

This configuration file, written as a JSON file, should contain
all the information needed about the database for which the application shall run. %information is singuler
For example it should contain the data bases name, like ``Skyserver'',  all the
dimensions of the data warehouse schema for it and all the measures. Also it
tells the parser, how to parse the lines in the log files.


Starting the application a \textit{Facade} instance will be constructed and initialized
with the configuration file. This instance will have mediators, 
\textit{ParserMediator} for the parsing taks, \textit{ChartMediator} for chart requests
and \textit{DataMediator} for things concerning the data access tier, all with a reference
to the configuration. So the methods in the 3 main section all can make their work depending
on the configuration. 

So how does a configuration look like? % thas --> does

\subsection{The configuration package}

\subsubsection{ConfigWrap} 
The class \textit{ConfigWrap} is the most important class concerning the config.
It is the class which wraps all the information taken from the configuratin file.
This contains a orderd array of \textit{DimRow}s and \textit{RowEntry}s.
 
It provides mostly getters for all the information, like number of dimensions,
a dimension or row at a certain index and so on.

\subsubsection{DimRow}
A DimRow stores the information whether at this point is a dimension or just a normal row (or measure).
For dimensions it provides methods to get information about the number of rows, a row at a index,
name of the table in the warehouse for this information, name of a row in the warehouse, 
a tree of strings for the web page to display in the selection boxes.


\subsubsection{RowEntry}
A RowEntry stores the information of on part in the parsing file. The subclasses of RowEntry
tell the parser, whether is a int, a String or something specific like location to parse. Therefore
they provide strategy methods.


\subsubsection{Addition note}
Why is everywhere written row, when it is a column in the warehouse? Yeah in the warehouse
it is a column, but in the schema of the warehouse it is a row\ldots




\subsection{Chart creation}
The visitor patter for the charts got dropped. Because facing some problems with \textit{ResultSet}
the decision was made to change the received data nearly to the warehouse into a JSONObject
and than add additional information later some way more up.

The design of three diffrent types of charts, with one, two or three dimensions was detected to
be stupid. There are no charts with 3 dimensions (on our web page). They either have one dimension and a measure
or two dimensions and a measure.\\
So \textit{DimChart} is now the parent class storing all information every chart needs, like
the first dimension and the measure. The child class \textit{TwoDimChart} extends this with a second dimension.

\subsubsection{ChartMediator}
The \textit{ChartMediator} is the mediator of a chart process. He creates a chart wrap
(\textit{DimChart}) with the helper class \textit{ChartHelper}
and mainly it delegates the requests against the data tier.






































\subsection{Parser}

\subsubsection{ParserMediator}

\texttt{threadPool} : Added a variable poolsize and a method \texttt{setPoolsize()} to set it to another number. 
If the poolsize is smaller than 1 there will be an error displayed. \newline\newline

\texttt{readyQueue , addToDatabase()} : Deleted, because the ParsingTask sends the finished dataEntry-Object directly to the DataTier.
\newline

\texttt{entryBuffer , extractLine()} : Deleted, because the ParsingTask requests a new line as soon as it needs it.\newline

\texttt{configDB} : Renamed to ConfigWrap.\newline

\texttt{parseLogFile(String str)} : Deleted configDB, because ParsingTask knows about the ParserMediator and can request it by itself.
\newline\newline

\texttt{error(String str), boolean fatalError} : Added a new error() method, which sets fatalError to true, which will 
make parseLogFile(String str) return false to the Facade.\newline

\texttt{increaseFT(), increaseLinedel()} : Added new methods to count finished threads and deleted lines. \newline

\texttt{createThreadPool()} : Creates a new threadPool with \textit{poolsize} threads and starts it. \newline

\texttt{ParserMediator(ConfigWrap cw)} : Creates the parserMediator.

\subsubsection{LogFile} 

\texttt{readLine()} : readLine() had bigger problems than expected, because the SQL-statement may contain several endOfLines. it's now 
fixed by setting a mark, reading another line, looking if it is actually a new line and if it is, resetting the mark and returning
the complete line with the whole statement. \newline

\texttt{Logfile(String path, ParserMediator pm)} : Checks if path is actually a valid csv-file and calls the Verification tool to check the
correctness of the formatting in the file.

\subsubsection{Task}

Deleted, because there are only ParsingTasks, no LoadingTasks and so there is no need for a superclass.

\subsubsection{LoadingTask}

Deleted, because the parsingTask sends the DataEntry directly to the Data Tier making the LoadingTask obsolete.

\subsubsection{ParsingTask}

\texttt{ParsingTask(ParserMediator pm)} : Creates a new ParsingTask.\newline

\texttt{splitStr} : Added the splitted string to the attributes of the ParsingTask so that it doesn't have to be splitted by every tool.

\subsubsection{DataEntry}

The DataEntry was completely remade to improve the general usability and multifunctionality. It now has only an Object-Array of variable size
which is filled due to the Config.

\subsubsection{VerifyTool} 

Renamed to VerificationTool. \newline

Completely remade the Verification tool, because the verify-part is already done by the splitting tool. Instead it checks, if the 
file is actually in the format from the config.

\subsubsection{LocationTool}

Renamed to GeoIpTool\newline

\textit{setUpIpTool(ParserMediator pm)} : Sets the GeoIPTool up and checks if the geoLiteCity.dat - file which is needed for this tool
is at the correct position.

\subsubsection{SplittingTool}

Splitted \texttt{split(ParsingTask pt)} in 3 parts to create smaller methods.

