\section{Changes in web tier}

The WebTier provides the graphical user interface and front end of our application.
On the client side it relies heavily on Javascript and the jQuery library.
As I haven't done much programming aside of the tasks given at university, which were solely in java and lately some C,
programming in Javascript was a very enlightening experience.

\subsubsection{Web page}
As for the index page, it has not changed very much on the surface since the prototype layout.
There is now an admin login and you can actually change the languages.
The play framework made some things, like form validation really easy and allowed us to program our application logic in java.
But as I had to learn the difference between scala and java is not as little as one may assume.
And because Play is mainly written in Scala it has only a rudimentary java documentation. That may be a problem, if you want
to access some advanced functions.
Because the specification of the web page was very vaguely, as we didn't know to which extend the front end would be
programmed in java or just plain html/javascript, I can't really state much that has changed since the specification.

\subsubsection{WebpageControllers}
This is the class Website: It provides a method for every valid url of the application, which is invoked when a user sends
a request.
There are many more additonal methods:
for login and form validation, chart request handling and so on.
the changeLanguage method now changes the language itself instead of calling the Localize class.
It still renders the HTML views. 'Render' because play provides the possibility to write scala code directly in this views.
These scala statements are evaluated at runtime when the page in question is requested.
Play can also serve static files. For that no involvement of a user defined controller is needed.

\subsubsection{Localize}
This class is still handling the localization of the webpage. in addition it localizes, other parts of the application,
e.g. status and error messages from the parser
the get() method was designed for the webpage. If a localized String can't be found, there is an automatic fallback to a
predefined standard language, so that still some information is displayed.
In the application itself that sometimes was not wanted behaviour, therefore another method has been added
getString, which can localize the same key words, but without the fallback mechanism.
The changeLanguage method has been removed from this class, as the Webpage controller can handle it itself.

\subsubsection{ChartIndex}
This class still scans the charts directory for all available charts at instanciation.
It is implemeted with the singleton pattern, to prevent unnecessary cpu and I/O time.
It's methods are still the same.

\subsubsection{what}
Some helper classes have been put in the what package although they are utilized by the WebpageController.
AdminLogin and LogfileUpload are two helper classes which facilitate form validation through usage of
functionality provided by Play.
AdminAuth helps in determining if a user is authenticated as admin or not and controlling the behaviour
of the secured section.
ChartHelper uses private methods to create chart pages dynamically and provides some caching mechanism
of these chart pages. The only public method is a static getOptions(String) which handles the singleton pattern
for ChartHelper and returns a Html contentType with all options which can be chosen to request the given chart.


 