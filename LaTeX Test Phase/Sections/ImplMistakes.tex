\section{Implementation mistakes}

Task determination problem

\subsubsection{Negative or too high poolsize}
The parser works with a pool of threads, which are created when parsing. In the ParserMediator there is
a variable poolsize, which is the used poolsize for parsing.
When using a negative or a really high number (tried Integer.MAX_VALUE), the program crashes.
So we implemented a check for the poolsize, which looks if it is between 1 and 50 and stops parsing if it isn't.

\subsubsection{NullPointerExceptions when correct file doesn't exist}
sp_parser.Logfile had a check, if the file which is given to it is actually existent. It used the file-class from java
and got a NullPointerException, which was caught and used as an indicator, that there is no file for the correct name.

This worked, but to improve our program, we used a class called FileHelper, which creates the file-object one time and
returns it when needed, because other parts of the program need the file too. It didn't return a NullPointerException when 
the file doesn't exist, so this check got overrun and the parser "thought" that there is a correct file which wasn't.
This caused the program to crash. Fixed by checking for null instead of catching a NullPointerException

\subsubsection{.trim()}
In some cases the logfile contains unneeded whitespaces, which produced false data. This was fixed by adding the .trim()-command
in StringRow.split() and StringMapRow.split()

\subsubsection{Anonymous Proxy}
Sometimes the city- and country-name of an IP returned Anynomous Proxy, which is now replaced by "other" to fit in with 
the other undefined countries and cities.


\ldots
