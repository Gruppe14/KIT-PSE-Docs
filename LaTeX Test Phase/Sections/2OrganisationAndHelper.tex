\section{Facade, Helper}
The Facade and utility classes are placed in the highest package of the application tier.

\subsection{Utility classes}\label{utility}
To bundle some functions needed in more than one package some utility
classes were created. They also made error handling and 
testing easier.

\paragraph{Change detail}
\begin{itemize}
  \item New utility classes created
\end{itemize}

This classes are described below.

\subsubsection{FileHelper}
Reading configuration files and reading log files share methods.
This includes getting the file,checking its existence,
rights and file extension. These functions
were collected in the class \textit{FileHelper}.


\subsubsection{JSONReader}\label{reader}
The configuration and parts of the communication between web tier
and data tier is accomplished with JSON objects. Also reading and
extracting from them is bound with possible exceptions and try-catch blocks.
To reduce duplicate code and provide easier access the \textit{JSONReader} class
was created.

\subsubsection{Printer}
The class \textit{Printer} was created to unify error, failure, problem, test
and success messages.


\subsection{Facade}
The facade's main functions are initializing everything and then direct incoming requests to
the specific mediators.

One problem found was that the creation of tables for the database caused serveral problems
for testing and when they changed. The creation of them was solved dynamically- so
that one just had to drop the old tables and when the system started, they would be created automatically.

\paragraph{Change detail}
\begin{itemize}
  \item Automatic check of the table existence and their creation.
\end{itemize}


\subsection{Testing}
The facade is tested via its use. If it wouldn't work or show any problems
nothing between web tier and application tier would work.

The Printer works fine.

The other classes have a JUnit test class.
\paragraph{Test details}
\begin{itemize}
  \item test FileHelper with wrong file and nonexisting files
  \item test whether needed Files for the system exist via FileHelper
  \item test all methods of JSONReader with correct keys
  \item test all methods of JSONReader with incorrect keys
\end{itemize}




