\section{Chart requests}

% integrity

% create query by themself


\subsection{Object oriented programming}
The original design for this package was dropped in the implementation phase.
There was neither a visitor pattern nor classes to wrap the information
needed efficiently. This decision was withdrawn.

\subsubsection{Chart request workflow} \label{chart}
The parameter lists in the methods got longer 
and longer when implementing the missing filter features
and the charts with two dimensions.

Something we didn't want to do at first, solved the problem. The problem 
with the \textit{ResultSet} closing, when closing the statement, made it
necessary to write the \textit{JSONObject} of the charts in the
data access tier. So the decision was made to send the whole chart instance
down to the data access tier. 

\paragraph{Change detail}
\begin{itemize}
  \item Chart itself is parameter of chart requests
\end{itemize}

Instead of requesting all the information now from the chart
and create the MySQL queries, this got changed to a more  elegant way.
The chart itself returns query parts for the things like \texttt{SELECT, JOIN} or other.

\paragraph{Change detail}
\begin{itemize}
  \item Creating of query parts in the chart
\end{itemize}

But it remained a bulge of information stored improperly. This was
solved with some classes described below.

\subsubsection{Filter class} 
Instead of requesting all query parts from the config, like table names and keys,
a filter stores the dimension on which it is based. First just the lowest
level of the filter values was stored, but this was detected to be incorrect.
A database may be child of more than one server and a city with the name 'x'
may be city in more than one country. For correct filtering, referring
to the things selected on the web page, trees were made for this values 
to build correct filter query parts.

\paragraph{Change detail}
\begin{itemize}
  \item \textit{Filter} class to wrap information of a filter
  \item trees of filter values instead of sets without parents
\end{itemize}


\subsubsection{Measure class}
The approach to store just a string for the measure was neither good style
nor was it possible to have other aggregations than \texttt{count(*)}.
This was solved by the new class \textit{Measure}. It made it also possible
to write more information in the JSON sent back to the web page about what
the returned chart is.

\paragraph{Change detail}
\begin{itemize}
  \item \textit{Measure} class to wrap information of a measure
\end{itemize}

\subsection{SQL injection}\label{inj}
It could have been possible to run SQL injections against the database
by using filter values as injection holes. Comparing the values of filters
parsed from a char request JSON with the actual values in the warehouse
should have closed this hole.

\paragraph{Change detail}
\begin{itemize}
  \item Protection against SQL injection
\end{itemize}


\subsection{Two axis of same dimension}
By playing with the chart we detected that queries for
charts with x- and y-axis of the same dimension incorrect
queries were created, resulting in not getting a chart.
\paragraph{Change detail}
\begin{itemize}
  \item Make it possible to create charts with x and y of same dimension
\end{itemize}



\subsection{Testing}

Correct reading of the JSON-Objects was tested by explicte use of filters, axis and measures, 
then printing the requests and comparing with the resulting MySQL queries. We couldn't see
the usage for any automatic tests there.

Creating the queries from the chart host was tested in the same way. 

\paragraph{Test details}
\begin{itemize}
  \item Print read data and check for correctness corresponding to web page request
  \item Print queries and check correctness
\end{itemize}
