\section{Web}
\subsection{web package}
The package web was created and the packages controllers and views have been moved there to seperate everything that is involved with the webpage and the frontend from the application logic in the what package.

\paragraph{Change detail}
\begin{itemize}
  \item added web package, moved views and controllers package to web package.
\end{itemize}

\subsection{Chart history}
One problem in displaying the history of the last requests, was to ensure that if an overview was sent to the user and he decided to check one out,
 that this chart history was still saved on the server side, even though there may have been new chart requests in the mean time.
To solve this problem the class ChartHelper was added. It generates an overview of the last chart requests, 
if a user requests it. In the process it saves the json Objects together with a unique user id (uuid). 
If the user requests a history, the requested history number will be looked up for his uuid and returned.

\paragraph{Change detail}
\begin{itemize}
  \item added ChartHelper which manages chart history overviews and requests.
\end{itemize}

\subsection{Possibility to add new chart types}
A chart consists of an thumbnail for the index page, a configuration file, which saves how many dimensions can be displayed by this chart type,
a javascript file, which contains the d3js code to visualize the data and an optional stylesheet file to style the chart elements.
To add new charts, you only have to add these files.
But to ensure that no chart types are displayed, which do not have all this files, their existence is now checked on application start.
For this purpose some methods were added and changed. There are methods to get all valid charts, to get the thumbnail name of a chart,
to get the number of dimensions a chart can display and to check if a chart has a stylesheet file or not.

\paragraph{Change detail}
\begin{itemize}
  \item changed and added methods.
\end{itemize}

\subsection{Reuse code in charts}
The code of different charts has to be independent, to ensure that additional charts can be added dynamically.
But then we wanted to reuse the code from bubblechart&scatterplot in the scatterplot, because they are identical except for the bubble radius. 
The solution was, that the javascript from scatterplot adds the javascript and stylesheet files from bubblechart&scatterplot dynamically itself. 
No serverside hardcoding involved. While this is no perfect solution, e.g. if the bubblechart&scatterplot is removed,
it fulfills it's purpose and allows the webserver to still handle all chart types and its files the same way.

\subsection{ChartHelper class}
Although the play framework template engine can call java functions, it was not dynamic enough to design our chart pages solely with its functions.
So the ChartHelper class was added, which creates the selection boxes for all charts in every supported language, 
with all selectable options, which are contained in the warehouse.
These are then saved as Html objects, so they don't have to be created on each call to a chart page.
The selection for a specific chart can be requested with the getOptions() method and it returns the selections in the right language,
as extracted from the user's session.

\paragraph{Change detail}
\begin{itemize}
  \item added ChartHelper class with methods to create option selections for charts.
\end{itemize}

\subsection{languages}
The playframe work itself supports UTF-8 encoded strings, so different languages shouldn't had been a problem. 
But as we had to notice the localisation didn't work for languages that don't use the latin alphabet.
This was a problem with the java property files and PropertyResourceBundle, where the localized strings are saved and which natively only
supports ISO-8859-1 encoding.
To solve this problem the class UTF8Control was added.

\paragraph{Change detail}
\begin{itemize}
  \item added web.controllers.UTF8Control, which allows reading UTF-8 encoded property files.
\end{itemize}

\subsection{Testing}
The testing of the webpage was done by hand as it made no sense to us, to test a webpage, 
the purpose of which is to display things graphically, in a nongraphical, automatic way.
Additionally this decision was enhanced by the fact that there are relatively few functions on the webpage to test.
The webpage has been tested in Chrome and Firefox, but additionally at least the main functions should also be available in Chromium,
Internet Explorer 10 and even on some mobile devices.
