\section{Configuration file}
This section is about the part of the system reading the configuration file
and handling its information.

\subsection{Requirements}
We decided that every configuration file and the order of the 
rows have to follow these requirements:
\begin{itemize}
  \item starts with 6 time rows and time rows contain the String 'time'
  \item has an IP row which follows time
  \item any other rows, which must be dimension (level > 0) if they are not of
  type integer or double
  \item rows of one dimensions follow each other with rising level
  \item integer or double rows are automatically measures
  \item at least 1 measure
  \item statement is at the last position
\end{itemize} 

For configurations following this requirements the system should run
successfully now. Some of the things are checked, some would cause a crash
of the build anyway.

\paragraph{Change detail}
\begin{itemize}
  \item Check the requirements to a configuration file
\end{itemize}

\subsection{Creation}
The reading of the configuration file was made easier with the
\textit{JSONReader} (\ref{reader}). Because of this the easy creation
of the \textit{ConfigWrap} was placed in the \textit{ConfigWrap} itself as
a static factory without the need of any helper class.

\paragraph{Change detail}
\begin{itemize}
  \item Static factory machine for \textit{ConfigWrap} without special helper class
\end{itemize}

\subsubsection{Unneeded attributes}
There were unneeded and senseless attributes for all rows in the configuration.
They got deleted and replaced by 'N/A' in the reading process.

\paragraph{Change detail}
\begin{itemize}
  \item Set deleted and unneeded row attributes to 'N/A'
\end{itemize}

\subsubsection{Space in names}
It was not possible to have spaces in names of rows, dimensions or a configurations. This
produced incorrect table and column names in the database.
\paragraph{Change detail}
\begin{itemize}
  \item Allow row, dimension and configuration names with space
\end{itemize}

\subsection{StringRow}
The \textit{StringRow} had uneeded functions. The idea was
to store all possible Strings of a row in it. But not really knowing what
could be content of the log files, the Strings are read directly from 
the log files and stored in the warehouse. So the old functions got obsolete.

\paragraph{Change detail}
\begin{itemize}
  \item Remove uneeded functions of \textit{StringRow} %ich habe kill erstetzt
\end{itemize}

\subsection{Dimension content trees}
The actual existing content of a String dimension is requested from
the warehouse and stored. This was made with a construct
of \textit{HashMap}s and \textit{TreeSet}s of String containing itself or not. It was
a bad approach to build a tree.

This was changed with a new class \textit{DimKnot} which forms tree structures. 
Instances of it also contain information about the level they are on,
which makes it much easier to build MySQL queries dynamically from them.
\paragraph{Change detail}
\begin{itemize}
  \item \textit{DimKnot} tree structure instead of bad \textit{HashMap} + \textit{TreeSet} approach
\end{itemize}



\subsection{Child dimensions}
All \textit{DimRow}s had a field for trees and time interval. To make this more
efficient and object oriented it was changed
by adding to child classes of \textit{DimRow}.
\paragraph{Change detail}
\begin{itemize}
  \item Child classes \textit{TimeDimension} and \textit{StringDim} for \textit{DimRow}
\end{itemize}

\subsection{Testing}

The configuration file and the test configurations where parsed and printed. 
This seems test enough that it is read correctly. 

The trees of the \textit{DimKnot}s where tested by checking what values are actually in the
database and if they all appear on the web page.

\paragraph{Test details}
\begin{itemize}
  \item Two new test configurations
  \item Parsing with 2 valid test configurations
  \item Look at web page after parsing if everythin fits too
  \item Parsing with invalid test configurations
\end{itemize}


