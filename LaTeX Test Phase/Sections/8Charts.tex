\section{Charts}
The charts saw a fair amount of bug fixes and improvements as well.

%\subsection{Bubblechart}
%A one dimensional bubblechart, where the items are sorted according to one dimension was implemented.
%TODO: War es schon immer da? :-)

\subsection{Bug fixes}
  \subsubsection{Type of data}
  The SkyServer contained a significant amount of categorical data. However, their representation on the axes
  used to cause problems. The charts are now more robust and work with both numerical and categorical data.
  
 \paragraph{Change detail}
\begin{itemize}
  \item Make charts robust against varying data
\end{itemize}
  
 \subsubsection{Errors}
  Errors are now thrown to differentiate between the chart with no data, and the chart with an error,
  or incorrect data (which the user will hopefully never experience). This was a expecially a problem
  at the Bubblechart, where no axis are printed.
  
   \paragraph{Change detail}
\begin{itemize}
  \item Show error pictures to diverge them from empty data
\end{itemize}
  
  \subsubsection{Boring bugs}
  Boring bugs, such as the text on the data was off by a few pixels were fixed too.
  
\subsection{Aesthetics}
The charts were made prettier.
  
\paragraph{Change detail}
\begin{itemize}
  \item Tooltips, or text that appears once the mouse hovers over a data point was added to every chart
  \item Consistent font use
  \item The text now gets rotated on a number of cases to fit more datapoints. It also makes the charts prettier
  \item Change the css % sollte das vll wo anders hin?
\end{itemize}


\subsection{Special cases}
%TODO
TODO wie zB nicht viele eintraege in barchart
TODO kleine bubble charts, nur buchstaben anzeigen
falls dir noch was einfaellt

\paragraph{Change detail}
\begin{itemize}
  \item Show a few amount of bars correct in barchart
  \item Just show sinle or no letters for small bubbles in bubblechart
\end{itemize}
  
\subsection{Testing}
 We tested by hand. Something that made testing in this section easier is that it is possible to know
 how a chart is suppose to look when correct, as well as the charts consisting of different more or less discrete
 parts - for example, it is easy to tell when just an axis is broken.
 Also, it is nice to return to a chart to see how nice someone made the css this time. 
 
 We also let some project extern person test and use the charts.

\paragraph{Test details}
\begin{itemize}
  \item Test by selecting axis, measure and filter combination
  \item Let project extern people use charts
\end{itemize}

