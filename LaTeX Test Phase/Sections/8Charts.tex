\section{Charts}
The charts saw a fair amount of bug fixes and improvements as well.

%\subsection{Bubblechart}
%A one dimensional bubblechart, where the items are sorted according to one dimension was implemented.
%TODO: War es schon immer da? :-)

\subsection{Bug fixes}
  \subsubsection{Type of data}
  The SkyServer contained a significant amount of categorical data. However, their representation on the axes
  used to cause problems. The charts are now more robust and work with both numerical and categorical data.
  
 \paragraph{Change detail}
\begin{itemize}
  \item Make charts robust against various data
\end{itemize}
  
 \subsubsection{Errors}
  Errors are now thrown to differentiate between the chart with no data, and the chart with an error,
  or incorrect data (which the user will hopefully never experience). 
  %This was a expecially a problem   at the Bubblechart, where no axis are printed.
  
   \paragraph{Change detail}
\begin{itemize}
  \item Show error messages on the chart pages.
\end{itemize}
  
  \subsubsection{Boring bugs}
  Boring bugs, such as the text on the data was off by a few pixels were fixed too.
  Other bugs include 
  \begin{itemize}
<<<<<<< HEAD
  	\item Bar charts not working correctly with only few data points.
  	\item There wasn't enough space for all text labels on bubble charts.
  	\item There wasn't enough space for all the text labels on pie charts.
  	\item There wasn't enough space for the text labels on the axes of scatterplots.
  	\item And so on.
=======
  	\item Bar charts not working correctly with only few datapoints.
  	\item There not being enough space for all text labels of all sizes, especially on bubble and pie charts.
>>>>>>> 5ed6ddda05c005e87d301827e45f27e46b8f5ab9
  \end{itemize}
  
  \paragraph{Change detail}
  \begin{itemize}
  	\item Bar charts with few bars are shown correctly. 
  	\item The small labels aren't shown on pie charts (they are shown on mouseovers)
  	\item Only as much of the label that fits in each bubble is shown on bubble charts.
\end{itemize}
  


\subsection{Aesthetics}
The charts were made prettier.
  
\paragraph{Change detail}
\begin{itemize}
  \item Tooltips, or text that appears once the mouse hovers over a data point was added to every chart
  \item The same fonts were used consistently.
<<<<<<< HEAD
  \item The CSS were made more elaborate.
  \item The text now gets rotated on a number of cases to fit more data points. It also makes the charts prettier

=======
  \item The css was made more elaborate.
  \item The text now gets rotated on a number of cases to fit more data points. It also makes the charts prettier
>>>>>>> 5ed6ddda05c005e87d301827e45f27e46b8f5ab9
\end{itemize}

  
\subsection{Testing}
 We tested by hand. Something that made testing in this section easier is that it is possible to know
 how a chart is suppose to look when correct, as well as the charts consisting of different more or less discrete
 parts - for example, it is easy to tell when just an axis is broken.
 Also, it was always nice returning to a chart to see how nicely someone improved the CSS this time. 
 
 We also let a person external to the project test and use the charts.

\paragraph{Test details}
\begin{itemize}
  \item Test by selecting many different axis, measure and filter combinations
  \item Let project external people use charts
\end{itemize}

