\section{Parser}

\subsection{Task determination} 
When having too many threads, sometimes a thread stopped working. This caused the counter for finished tasks
to be stuck below the number of threads and the program not to get finished.
So a watchdogtimer was implemented which replaces a thread by a new one, if said thread doesn't finish a line for a 
predefinied amount of time.

\paragraph{Change detail}
\begin{itemize}
  \item Added watchdogtimer
\end{itemize}

This worked fine till other big problems showed up and no thread worked at all. This caused
a endless creation of new threads.

\paragraph{Change detail}
\begin{itemize}
  \item Stop endless creation of new threads
  \item Kill parsing request if to many threads fall asleep
\end{itemize}


\subsection{Negative or too high poolsize}
The parser works with a pool of threads, which are created when parsing. In the ParserMediator there is
a variable poolsize, which is the used poolsize for parsing.
When using a negative or a really high number (tried Integer.MAX\_VALUE), the program crashes. 
So we implemented a check for the poolsize, which looks if it is between 1 and 50 and stops parsing if it isn't.

\paragraph{Change detail}
\begin{itemize}
  \item Check for invalid poolsize before parsing.
\end{itemize}

\subsection{NullPointerExceptions when correct file doesn't exist}
sp\_parser.Logfile had a check, if the file which is given to it is actually existent. It used the file-class from java
and got a \texttt{NullPointerException}, which was caught and used as an indicator, that there is no file for the correct name.

This worked, but to improve our program, we used a class called \textit{FileHelper} (\ref{FileHelper}). 
It didn't return a \texttt{NullPointerException} when the file doesn't exist, 
so this check got overrun and the parser 'thought' that there is a correct file, which was not the case.
This caused the program to crash. 

\paragraph{Change detail}
\begin{itemize}
  \item Use and correct use of \textit{FileHelper}
\end{itemize}

\subsection{Whitespace in entries}
Some log files contain unneeded whitespace, which produced incorrect data. 
This was fixed by adding the \texttt{trim}-command in StringRow.split() and StringMapRow.split()

\paragraph{Change detail}
\begin{itemize}
  \item Trim the strings before sending them to the database.
\end{itemize}

\subsection{Anonymous Proxy}
GeoIp sometimes returned 'Anynomous Proxy' instead of nothing for cities or countries it didn't find.
This gets replaced by 'other' to fit in with the other undefined countries and cities.


\paragraph{Change detail}
\begin{itemize}
  \item Check for anynomous proxy
\end{itemize}

\subsection{Check for missing or empty parts}
The parser now checks whether a part of the line parsed is missing or empty. In this case the line is ignored and 
the count for incorrect lines gets incremented.

\paragraph{Change detail}
\begin{itemize}
  \item Check for missing or empty part in every line
  \item Flawed lines are not accepted
\end{itemize}

\subsection{Only accept when enough lines submitted}
Added \texttt{final static double CORRECT}, a double between 0 and 100, which indicates how many percent of the lines need to be
submitted correctly for the ParserMediator to return true, singaling parsing was successful.

\paragraph{Change detail}
\begin{itemize}
  \item Only accepts when more than a specific amount of lines are accepted.
\end{itemize}

\subsection{Reset}
The parser now resets after it finished parsing a logfile, so that another file can be parsed.

\paragraph{Change detail}
\begin{itemize}
  \item added a reset-method in the ParserMediator
\end{itemize}


\subsection{Testing}

Besides the JUnit tests, described below, there were some manual tests. One of those was parsing log files with
just a few lines into an empty database, and then comparing the result.

\paragraph{Test details}
\begin{itemize}

\item Mistakes 
\begin{itemize}
\item testIntegerMistake - year "2013apples" instead of "2013"
\item testMissingPart - with a missing statement
\item testEmptyPart - with one empty statement and one empty year
\item smallParseTestEmptyLine - with empty lines in between
\item flawedFile - try to parse from a file with a typo in "type" %-> Error \#11: The configuration file got a different format.
\end{itemize}

\item Poolsize
\begin{itemize}
\item smallParseTestSize10 - parse 10 lines, poolsize 10
\item smallParseTestTooBig - parse 10 lines, poolsize 100 %-> Error \#2: ParserMediator.poolsizeParsing is bigger than 10.
\item negativePoolsize - parse 10 lines, poolsize -2 %-> Error \#1: ParserMediator.poolsizeParsing is smaller than 1.
\item mediumParseTestSize50 - parse 1000 lines, poolsize 10
\end{itemize}

\item Just parsing
\begin{itemize}
\item smallParseTestStandardSize - parse 10 lines, standard (5) poolsize
\item mediumParseTestStandardSize - parse 1000 lines, standard (5) poolsize
\item bigParseTestStandardSize - parse 10k lines, standard (5) poolsize
\item doubleParsing - parse 1000 lines twice from seperate files
\end{itemize}

\item Ignored
\begin{itemize}
\item veryBigParseTestStandardSize (IGNORED) - parse a whole month, standard (5) poolsize, ignored because it may take hours 
to finish.
\end{itemize}

\end{itemize}
