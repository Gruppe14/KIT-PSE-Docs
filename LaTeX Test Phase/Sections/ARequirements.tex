\section{Requirements}\label{spec}


\subsection{Test cases}
It showed that the test cases specified in the functional specification
don't really gurantee proper work of the system. They just show
that the core criteria are fullfilled. 

So we made new and more specific tests as desribed in the
corresponding sections. With them the whole system should be covered by tests.

\subsection{Functional requirements}
It can be said that nearly all functional requirements are fullfilled.

One exclusion is the following. With the decision to make all work 
based on configurations this got obsolet. Also the names of dimensions
and measures are explaining itself enough.
\begin{itemize}
  \item /F110/ Show information about selectable variables
\end{itemize}

\subsection{Nonfunctional requirements}
\subsubsection{Usability}
The requirements for usability were the following. 
\begin{itemize}
  \item /N10/ Minimalistic GUI design
  \item /N20/ Responsive GUI
  \item /N30/ GUI that is easy to get used to
\end{itemize}
They are hard to test and subjective, but we think they are fullfilled
and the web page looks great.


\subsubsection{Swiftness}
Not using OLAP cubes the chart requests get slower and slower
with more data in the warehouse.

Changing the loading of \textit{DataEntry}s (\ref{noAu}) increased
the speed of parsing a lot. See the number of lines and the time in milliseconds
for the old and the new way.

\begin{tabular}{l|c|c}
\# of rows & old & no auto commit \\
\hline
10000& 14 017 & 13 016\\
10000&11 007 &9 014\\
10000&12 005&10 008\\
10000&11 007&9 012\\
10000&12 017&9 007\\
100000&105 009 &92 012\\
100000&106 009 &78 010\\
\end{tabular}

This showes a decrease of about 18\%.

Parsing a bigger file with 1146315 lines showed
that it takes about 11.23ms per line. (On a system with no music, webbrowser,
Eclipse and other programs running this may be faster.)