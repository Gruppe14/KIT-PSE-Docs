\section{Requirements}\label{spec}


\subsection{Test cases}
It showed that the test cases specified in the functional specification
don't really guarantee proper work of the system. They just show
that the core criteria are fulfilled. 

So we made new and more specific tests as described in the
corresponding sections. With them the whole system should be covered by tests.

\subsection{Functional requirements}
It can be said that nearly all functional requirements are fulfilled.

The exclusion being the following: /F110/ Show information about selectable variables
With the decision to make all work based on configuration files this got obsolete.%Did it? We can just write the descriptions in the configuration file
On the other hand, from a pragmatical standpoint, the names of the dimensions
and measures are pretty expressive and self explanatory. % and are all used in the same way.
\begin{itemize}
  \item /F110/ Show information about selectable variables
\end{itemize}

\subsection{Nonfunctional requirements}
\subsubsection{Usability}
The requirements for usability were the following. 
\begin{itemize}
  \item /N10/ Minimalistic GUI design
  \item /N20/ Responsive GUI
  \item /N30/ GUI that is easy to get used to
\end{itemize}
These are hard to quantify and are to some extend subjective, 
our highly biased opinion being that the webpage looks great and that they are fulfilled.

However
\begin{itemize}
	\item The website has barely not functionality except the needed one.
	\item From a navigational aspect, it isn't any different than a normal website. (Thus if a user can use a normal website they can use our program, too.)
\end{itemize}



\subsubsection{Swiftness}
Not using OLAP cubes the chart requests get slower with more data in the warehouse. %Really?

Changing the loading of \textit{DataEntry}s (\ref{noAu}) increased
the speed of parsing a lot. See the number of lines and the time in milliseconds
for the old and the new way.

\begin{tabular}{l|c|c}
\# of rows & old & no auto commit \\
\hline
10000& 14 017 & 13 016\\
10000&11 007 &9 014\\
10000&12 005&10 008\\
10000&11 007&9 012\\
10000&12 017&9 007\\
100000&105 009 &92 012\\
100000&106 009 &78 010\\
\end{tabular}

This shows a decrease of about 18\%.

Parsing a bigger file with 1146315 lines showed
that it takes about 11.23ms per line. (On a system with no music, web browser,
Eclipse and other programs running this may be faster.)
%I don't think we should mention that. It just shows that we failed to make a good test