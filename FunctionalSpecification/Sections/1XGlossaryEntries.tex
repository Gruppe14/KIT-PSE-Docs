\newglossaryentry{SkyServer}
{
  name=SkyServer,
  description={is a big \gls{database} for astronomical data. It contains pictures
              and other information concerning astronomy and tries to 'form a 
              map of the universe'.
              }
}


\newglossaryentry{KIT}
{
  name=KIT,
  plural=KIT,
  description={is a technological university in Karlsruhe, Baden-Württemberg, Germany 
              (Karlsruhe Institute of Technology).}
}


\newglossaryentry{log file}
{
  name=log file,
  description={
% Think about beind a child, doing something, then telling your mum. 
% Logging is telling your mum about something you did, after you did it. While using a computer, and without
% ensuring that your grandchildren(*) will know about your embarassing childhood.
% also, http://en.wikipedia.org/wiki/Journal_(computing)
The server log file records information about \glspl{query} run against a  \gls{database}.}
}


\newglossaryentry{.csv}
{
  name=.csv,
  description={.csv is a file format which is used for log files, 
  saving them in a standardized formatting schema.}
}


\newglossaryentry{database}
{
  name=database,
  plural=databases,
  description={A database is a structured collection of data. Its main task is
  storing a huge amount of data efficiently, without inconsistency and over a long period of time.   
  \href{https://en.wikipedia.org/wiki/Database}{(Wiki)}}
}


\newglossaryentry{PSE}
{
  name=PSE,
  plural=PSE,
  description={PSE is short for Praxis der Software-Entwicklung (engl. Software Engineering Practice).
  It is part of the computer science bachelor at the \gls{KIT}.}
}


  \newglossaryentry{data warehouse}
{
  name=data warehouse,
  plural=data warehouse,
  description={A data warehouse is a \gls{database} used for reporting and data analysis.
  It is a central repository of data which is created by integrating data from multiple disparate sources.
  \href{https://en.wikipedia.org/wiki/Data_warehouse}{(Wiki)}}
}
  
  
  \newglossaryentry{star schema}
{
  name=star schema,
  plural=star schema,
  description={The star schema is a simple style of a \gls{data warehouse} schema.
 \href{https://en.wikipedia.org/wiki/Star_schema}{(Wiki)}}
}

  
  \newglossaryentry{query}
{
  name=query,
  plural=queries,
  description={A query is a request to a \gls{database} or \gls{data warehouse}.}
}


  \newglossaryentry{parser}
{
  name=parser,
  plural=parser,
  description={A parser describes the program, which just has the function of analyzing a text, 
  to determine its grammatical structure with respect to a given formal grammar. Or in other words,
  it extracts parts of a text respectivly to their function or meaning. 
  \href{https://en.wikipedia.org/wiki/Parser}{(Wiki)} }
}


  \newglossaryentry{GUI}
{
  name=GUI,
  plural=GUI,
  description={A graphical user interface, short GUI, is a type of user interface 
  that allows users to interact with electronic devices using images rather 
  than text commands. \href{https://en.wikipedia.org/wiki/Graphical_user_interface}{(Wiki)}}
}


  \newglossaryentry{diagram}
{
  name=diagram,
  plural=diagrams,
  description={A diagram is a symbolic representation of information according to some
  visualization technique. The term diagram is used with the same meaining as \gls{chart}.
  \href{https://en.wikipedia.org/wiki/Diagram}{(Wiki)}}
}


  \newglossaryentry{chart}
{
  name=chart,
  plural=charts,
  description={Used with the same meaning as \gls{diagram}.}
}


  \newglossaryentry{dimension}
{
  name=dimension,
  plural=dimensions,
  description={A dimension describes either a dimension of a \gls{diagram} or a dimension in a
  \gls{data warehouse}.  }
}


  \newglossaryentry{ETL-process}
{
  name=ETL-process,
  plural=ETL-process,
  description={ETL - extract, transform, load - refers to a process in \gls{data warehouse} usage. This involves
  extracting data from the outside, transforming it to fit operational needs and loading it into the end target.
 \href{https://de.wikipedia.org/wiki/ETL-Prozess}{(Wiki)}}
}


  \newglossaryentry{SQL}
{
  name=SQL,
  plural=SQL,
  description={SQL -Structured Query Language - is a special-purpose programming language designed for 
  managing data in relational \gls{database} management systems. \href{https://en.wikipedia.org/wiki/Sql}{(Wiki)}}
}


  \newglossaryentry{Java VM}
{
  name=Java VM,
  plural=Java VM,
  description={A Java virtual machine (JVM) is a virtual machine that can execute Java bytecode. 
  \href{https://en.wikipedia.org/wiki/Java_virtual_machine}{(Wiki)}}
}


 \newglossaryentry{histogram}
{
  name=histogram,
  plural=histograms,
  description={A histogram is a graphical representation showing a visual impression of the distribution of data. 
   \href{http://en.wikipedia.org/wiki/Histogram}{(Wiki)}}
}


 \newglossaryentry{bubble chart}
{
  name=bubble chart,
  plural=bubble charts,
  description={A bubble chart is a type of chart that displays three 
  dimensions of data. Two values are expressed through the disk's xy location and the third 
  through its size. Bubble charts can facilitate the understanding of social,
economical, medical, and other scientific relationships. \href{http://en.wikipedia.org/wiki/Bubble_chart}{(Wiki)}}
}


 \newglossaryentry{scatter plot}
{
  name=scatter plot,
  plural=scatter plots,
  description={A scatter plot or scattergraph is a type of mathematical diagram. 
  The data is displayed as a collection of points, each having the value of one 
  variable determining the position on the horizontal axis and the value of the other 
  variable determining the position on the vertical axis. \href{http://en.wikipedia.org/wiki/Scatter_plot}{(Wiki)}}
}


   \newglossaryentry{measure}
{
  name=measure,
  plural=measures,
  description={A measure is a variable in a model with a specific value range.}
}
  \newglossaryentry{WHERE-part}
{
  name=WHERE-part,
  plural=WHERE-part,
  description={The \texttt{WHERE}-part is a part of a \gls{query}. It includes a comparison predicate, 
  which restricts the \glspl{row} returned by the \gls{query}. 
  The WHERE clause eliminates all rows from the result set 
  for which the comparison predicate does not evaluate to True.}
}


 \newglossaryentry{row}
{
  name=row,
  plural=rows,
  description={In the context of a \gls{database}, a row represents a single, implicitly 
  structured data item in a table. Each row in a table represents a set of related data, 
  and every row in the table has the same structure.
  \href{(http://en.wikipedia.org/wiki/Row_(database))}{(Wiki)}}
}


   \newglossaryentry{busy time}
{
  name=busy time,
  description={Busy time can also be refered to as CPU time and is the amount of time which a 
  central processing unit (CPU), in our case the \gls{SkyServer}, needed to process instructions,
   e.g. to serve the requested data. \href{(http://en.wikipedia.org/wiki/CPU_time)}{(Wiki)}}
}


 \newglossaryentry{elapsed time}
{
  name=elapsed time,
  description={Elapsed (real) time is the time taken from start of computer program to the end. 
  Elapsed real time includes I/O time and all other types of wait. 
  \href{(http://en.wikipedia.org/wiki/Elapsed_real_time)}{(Wiki)}}
}


\newglossaryentry{browser}
{
  name=browser,
  description={A piece of software mostly used to navigate the enchanting world of the internet, and thus most commonly
used to watch illegal movies and creep your neighbours photos on facebook. Can be also used to view nice charts.}
}


\newglossaryentry{Firefox}
{
  name=firefox,
  description={A browser that is both modern and popular. Other browsers are popular too, but not modern. }
}


\newglossaryentry{Chrome}
{
  name=chrome,
  description={Chrome is a modern browser like firefox, with the notably difference being that google now knows about the
creeping of your neighbour. See browser.}
}

