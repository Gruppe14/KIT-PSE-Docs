\section{Goals}

%=======================================================================

%Through or with?
With this program the user should be put in the position 
to visualize the prepared data of queries, made against his data base.


To structure the criteria the system has to fulfill, 
it is divided into three parts for this and other sections.
\begin{itemize}
  \item Web page
  \item Parser
  \item Analyzer
\end{itemize}
See \ref{overview} for a overview of this parts and their relationships.
In the following their specific goals and criteria are described.
 

% \subsection{Core criteria}
% This criteria have to be fulfilled to guarantee 
% the core functionality of the system.
% 
% \begin{itemize}
%   \item Provide an option to import CSV-formatted log files % glossary
%    and parse specified data out of them
%   \item Load 
% \end{itemize}


%=======================================================================
\subsection{Web page}
\subsubsection{Core criteria}
\begin{itemize}
\item The web page is the graphical interface for users and administrators. 

\item Users can choose the charts they want to see 
and also choose and filter the dimensions and measures for those. 
It has to support at least scatter plots, histograms and bubble charts.

\item The web page provides help for how to use the diagrams and variables (dimensions 
and measures). 
Also the navigation on the page is easy to use and straight forward.

\item A admin-login provides administrative rights, which are needed to use the parser.
\end{itemize}

%----------------------------------------------------------------------
\subsubsection{Optional criteria}
\begin{itemize}
\item The language of the website can be changed into different languages. (e.g. German)

\item The administrator gets the ability to handle the data warehouse and load new log-files.

\item A little history of the last requested charts is stored and viewable.
\end{itemize}

%----------------------------------------------------------------------
\subsubsection{Exclusion criteria}
\begin{itemize}
\item The web page does not allow normal users to load new data into the data warehouse.

\item The web page does not provide statistic tabular. 
\end{itemize}


%=======================================================================
\subsection{Parser}

\subsubsection{Core criteria} % Musskriterien
\begin{itemize}
\item The Parser is able to perform the ETL-process on CSV-formatted log files (from Skyserver). %glossary
This means he extracts the data he needs from the files, transforms them 
and loads them into the data-warehouse. Dimensions and measures are specified in \ref{WHschema}.
  
\item The Parser will recognize invalid logs and won't add them to the data-warehouse.
 Every log with a mistake won't be accepted, because an error-message is not 
 as bad as a corrupted data-warehouse. 
 
\item The Parser will be fed with log files from the administrator via web page.
\end{itemize} 

% %----------------------------------------------------------------------
% \subsubsection{Optional criteria}
% \begin{itemize}
% \item 
% \end{itemize} 
 
%----------------------------------------------------------------------
\subsubsection{Exclusion criteria}
\begin{itemize}
\item This parser is only able to operate on CSV-formatted logs from SkyServer. 
It can neither read logs in another formats nor logs from another source.

\item There is no way to avoid using this Parser when adding data to the warehouse. 
This can stop corrupting the warehouse to guarantee correct data in the warehouse.

\item The Parser doesn't correct mistakes in the log file.
\end{itemize}



%=======================================================================
\subsection{Analyzer}

\subsubsection{Core criteria}
\begin{itemize}
\item The analyzer is the gate to the data warehouse. It extracts the specific data, 
needed for the diagrams, passing them to the java-script front-end.
\item The analyzer can take filtered information via web page from the user 
to use only certain data for the charts.
\item The charts that are supported are at least:
\begin{itemize}
\item scatter plots
\item histograms
\item bubble charts
\end{itemize}

\end{itemize}

%----------------------------------------------------------------------
\subsubsection{Optional criteria}
\begin{itemize}
\item The analyzer will support more different chart-types, especially the combination of
a histogram and a scatter plot. 
\item The analyzer does a little bit of data mining, and presents some potentially interesting information.
\end{itemize}

%----------------------------------------------------------------------
% \subsubsection{Exclusion criteria}
% \begin{itemize}
% \item 
% \end{itemize}


% Vll sollte man noch was zum warehouse sagen.



