\section{Goals}

%=======================================================================

\subsection{Parser}

% TODO generell goal specification

\subsubsection{Obligatory criteria} % Musskriterien
\begin{itemize}
\item The Parser can read a CSV-formatted log file from skyserver.sdss.org
 and fill a data-warehouse with multi-dimensional user information.
 The dimensions of this data are location of the user, location of the 
 database and time of the server-access. It will also add several measures such as delay time or  
\item The Parser will recognise invalid logs and won't add them to the data-warehouse.
 Every log with a mistake won't be accepted, because an error-message is not 
 as bad as a corrupted data-warehouse. 
\item The Parser will be accessible with admin-login to the webpage.
\end{itemize}

%----------------------------------------------------------------------
\subsubsection{Optional criteria}
\begin{itemize}
\item The parser will contain a small manual which is designed for experienced users. (optional because this parser is
only used by it's developers)
\end{itemize}

%----------------------------------------------------------------------
\subsubsection{Exclusion criteria}
\begin{itemize}
\item The Parser can only use CSV-formatted logs from skyserver.sdss.org. 
It can neither read logs in another format nor logs
from another source.
\item There is no way to avoid using this Parser when adding data to the warehouse. 
This can stop corrupting the warehouse to guarantee correct data in the warehouse.
\item The Parser doesn't correct mistakes in the log file.
\item Due to being only a tool for it's developers, our parser doesn't need to be easy to use for first-time users.
\end{itemize}



%=======================================================================
\subsection{Analyzer}

\subsubsection{Obligatory criteria}
\begin{itemize}
\item The analyzer can use the data-warehouse which is filled with user data  
from skyverser.sdss.org to create various charts. 
Those charts are created by user command and are visible on a javascript-website.
\begin{itemize}
\item scatterplots
\item histograms
\end{itemize}
\item The analyzer can take filter information via webpage from the user to use only certain data for creating charts.
\end{itemize}

%----------------------------------------------------------------------
\subsubsection{Optional criteria}
\begin{itemize}
\item The analyzer will support different chart-types.
\end{itemize}

%----------------------------------------------------------------------
\subsubsection{Exclusion criteria}
\begin{itemize}
\item WHAT takes correct data in the warehouse for granted. It doesn't check the
 data because it already checked by SLP. 
\end{itemize}


%=======================================================================
\subsection{Website}
\subsubsection{Obligatory criteria}
\begin{itemize}
\item The website has an graphical interface in javascript, so that users can enter which chart they want.
\item The website contains small guides for experienced users on how to create a specific chart. 
(similar to the guides at skyserver.sdss.org)
\end{itemize}
\item The website has an admin-login page to get administrative rights, which are needed to use the parser.

%----------------------------------------------------------------------
\subsubsection{Optional criteria}
\begin{itemize}
\item The language of the website can be changed into different languages. (e.g. German)
\end{itemize}

%----------------------------------------------------------------------
\subsubsection{Exclusion criteria}
\begin{itemize}
\item The website got no function for normal users to enter data into the data warehouse.
\end{itemize}

