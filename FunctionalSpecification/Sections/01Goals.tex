\section{Goals}

%=======================================================================

%Through or with?
With this program the user should be put in the position 
to visualize the prepared data of \glspl{query}, made into his data base.


To structure the criteria the system has to fulfill, 
it is divided into three parts for this and other sections.
\begin{itemize}
  \item Web page
  \item \gls{parser}
  \item Analyzer
\end{itemize}
See \ref{overview} for an overview of these parts and their relationships.
In the following section their specific goals and criteria are described.
 

% \subsection{Core criteria}
% This criteria have to be fulfilled to guarantee 
% the core functionality of the system.
% 
% \begin{itemize}
%   \item Provide an option to import CSV-formatted log files % glossary
%    and parse specified data out of them
%   \item Load 
% \end{itemize}


%=======================================================================
\subsection{Web page}
\subsubsection{Core criteria}
\begin{itemize}
\item The web page is the graphical user interface (\gls{GUI}) for users and administrators. 

\item Users can choose the charts they want to see 
and also choose and filter the dimensions and measures for those. 
It has to support at least \glspl{scatter plot}, \glspl{histogram} and \glspl{bubble chart}.

\item The web page provides help for how to use the \glspl{diagram} and variables (\glspl{dimension} 
and measures). 
Also the navigation on the page is easy to use and straight forward.

\item An admin-login provides administrative rights, which are needed to use the \gls{parser}.
\end{itemize}

%----------------------------------------------------------------------
\subsubsection{Optional criteria}
\begin{itemize}
\item The language of the web page can be changed (e.g. German).

\item The administrator gets the ability to handle the \gls{data warehouse} and load new \glspl{log file}.

\item A little history of the last requested charts is stored and viewable.
\end{itemize}

%----------------------------------------------------------------------
\subsubsection{Exclusion criteria}
\begin{itemize}
\item The web page does not allow normal users to load new data into the \gls{data warehouse}.

\item The web page does not provide statistical tables. 
\end{itemize}


%=======================================================================
\subsection{Parser}

\subsubsection{Core criteria} % Musskriterien
\begin{itemize}
\item The \gls{parser} is able to perform the \gls{ETL-process} on \gls{.csv} \glspl{log file}
 (from \gls{SkyServer}). 
This means it extracts the data it needs from the files, transforms them 
and loads them into the data-warehouse. \gls{dimension} and measures are specified in \ref{WHschema}.
  
\item The \gls{parser} will recognize invalid logs and won't add them to the \gls{data warehouse}.
 Every log with a mistake won't be accepted, because an error-message is not 
 as bad as a corrupted \gls{data warehouse}. 
 
\item The \gls{parser} will be fed with \glspl{log file} from the administrator via web page.
\end{itemize} 

% %----------------------------------------------------------------------
% \subsubsection{Optional criteria}
% \begin{itemize}
% \item 
% \end{itemize} 
 
%----------------------------------------------------------------------
\subsubsection{Exclusion criteria}
\begin{itemize}
\item This \gls{parser} is only able to operate on \gls{.csv} \glspl{log file} from \gls{SkyServer}. 
It can neither read logs in another formats nor logs from another source.

\item There is no way to avoid using this \gls{parser} when adding data to the \gls{data warehouse}. 
This prevents corrupting the \gls{data warehouse} and guarantees correct data in it.

\item The \gls{parser} doesn't correct mistakes in the \glspl{log file}.
\end{itemize}



%=======================================================================
\subsection{Analyzer}

\subsubsection{Core criteria}
\begin{itemize}
\item The analyzer is the gate to the data warehouse. It extracts the specific data, 
needed for the diagrams, passing them to the java-script front-end.
\item The analyzer can take information filtered by user-selected criteria, 
to use only certain data for the charts.
\item The charts that are supported are at least:
\begin{itemize}
\item \glspl{scatter plot}
\item \glspl{histogram}
\item \glspl{bubble chart}
\end{itemize}

\end{itemize}

%----------------------------------------------------------------------
\subsubsection{Optional criteria}
\begin{itemize}
\item The analyzer will support more chart-types, for example the combination of
a \gls{histogram} and a \gls{scatter plot}. 
\item The analyzer does a little bit of data mining, and presents some potentially interesting information.
\end{itemize}

%----------------------------------------------------------------------
% \subsubsection{Exclusion criteria}
% \begin{itemize}
% \item 
% \end{itemize}


% Vll sollte man noch was zum warehouse sagen.



