\section{Goals}

%=======================================================================

\subsection{Server Log Parser (SLP)}

% TODO generell goal specification

\subsubsection{Obligatory criteria} % Musskriterien
\begin{itemize}
\item The SLP can read a CSV-formatted log file from sdss.skyserver.org
 and fill a data-warehouse with multi-dimensional user information.
 The dimensions of this data are location of the user, location of the 
 database and time of the server-access. 
\item SLP will recognise invalid logs and won't add them to the data-warehouse.
 Every log with a mistake won't be accepted, because an error-message is not 
 as bad as a corrupted data-warehouse. 
\item SLP will be accessible with admin-login to the webpage.
\end{itemize}

%----------------------------------------------------------------------
\subsubsection{Optional criteria}
\begin{itemize}
\item SLP will contain a small manual which is designed for experienced users.
\end{itemize}

%----------------------------------------------------------------------
\subsubsection{Exclusion criteria}
\begin{itemize}
\item The SLP can only use CSV-formatted logs from sdss.skyserver.org. 
It can neither read logs in another format nor logs
from another source.
\item There is no way to avoid using SLP when adding data to the warehouse. 
This can stop corrupting the warehouse to guarantee correct data in the warehouse.
\item SLP doesn't correct mistakes in the log file.
\item Due to being only a tool for the persons who wrote it, SLP doesn't need to be easy to use for first-time users.
It will be a console program or maybe contain a simple GUI.
\end{itemize}



%=======================================================================
\subsection{Warehouse Analyzing Tool (WHAT)}

\subsubsection{Obligatory criteria}
\begin{itemize}
\item WHAT can use the data-warehouse which is filled with user data  
from sdss.skyverser.org to create various charts. 
Those charts are created by user command and are visible on a javascript-webpage.
\begin{itemize}
\item scatterplots
\item histograms
\end{itemize}
\item The website contains small guides for experienced users on how to create a specific chart. 
(similar to the guides at skyserver.sdss.org)
\end{itemize}

%----------------------------------------------------------------------
\subsubsection{Optional criteria}
\begin{itemize}
\item The language of the website can be changed into different languages. (e.g. German)
\item WHAT will support different chart-types.
\end{itemize}

%----------------------------------------------------------------------
\subsubsection{Exclusion criteria}
\begin{itemize}
\item WHAT takes correct data in the warehouse for granted. It doesn't check the
 data because it already checked by SLP. 
\end{itemize}
