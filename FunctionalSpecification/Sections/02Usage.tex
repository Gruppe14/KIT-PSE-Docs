\section{Usage}
%Relevant
% http://www.geekherocomic.com/comics/2009-02-06-prima-donna-developer.png
\subsection{Area of Application}
%Deutsche Wort: Anwendungsbereich <- ja klingt besser
The application area of this product is scientific \glspl{database} that need to be analysed, % <- 'is' wegen 'area'
 understood and visualised.
The program has been designed with the \gls{SkyServer} in mind,
 so that is the primary application area.
However, as the only major requirement is the existence of correct access logs of the \gls{database}.
The software may also be used with many other scientific \glspl{database},
 although minor modifications may be needed.


%RIP, piece of art
%Haha aggro
%Let's say you have some data in a database(who hasn't, right?).
%Let's say you don't feel like wasting your only life and what to see what parts of that data ara important, fast.
%Let's say you don't have the time, inclination, or even (gasp!) technical background to do so yourself.
%
%Fear not. WHAT is exactly what you are looking for, and more!
%With features such as drawing of scatterplots, histograms, bubble charts* 
%and other obscure charts you have never heard of before, statistics about the importance of your tables*,
%automatic reports of* of interesting correlations*, what is the right tool for your job!

%As a concrete implementation, all this will be provided for the SKYSERVER logs dataset, but other datasets may be
%used as well, with minor or not as minor modifications.

%(* indicates that this is not a core feature, may be included later in the project, or not at all.)

\subsection{Target groups}

The target group of this application is people that want to analyse 
and visualize queries made against a \gls{database}. Specifically,
in our case, the \gls{SkyServer}.

This includes
\begin{itemize}
  \item the owners of the \gls{database} who want to optimize the access to
  	their data,
  	
  \item people that are interested in what others use the \gls{database} for,

  %\item people knowing or using the database, wondering for what and when other people
 % use the database, %is Punkt 2
  
  %\item people new to the database, wanting to know, what may be of interest
 % on the database.
  \item people new to the \gls{database}, who want to understand it or discover interesting information.

\end{itemize}
To summarize, WHAT will be of interest for many people, 
whether they want to see how their \gls{database} is used, analyse current trends,
or just love statistics and \glspl{diagram}.

% \begin{enumerate}  
%   \item people that know about the skyserver. Since the skyserver only 
%    allows queries via sql and has an arcane website this greatly 
%    reduces the prospective audience. We do not expect the webserver 
%    to run into scaling issues.
%   
%   \item People that are interested in what other people are 
%   using the web server for. Our project mainly visualizes sql queries
%    from other people. Knowing sql, while not a prerequisite, 
%    would allow the user to fully utilize the software.
%   
% \end{enumerate}


%We expect a audience knowing what they want. This means,
%there should not be huge support choosing useful variables and scales for
%the charts. 

%Also we exspect adminstrators to be well versed with technical matters.

We assume a rather technical audience.
 
% That said, this does not prevent the (web) user interface from being functional,
% usable and prettier than what you would expect a group of 
% computer science students to design.
 

\subsection{Operating conditions}

The program is mainly used as a website, with the primary difference being
 that the server has to be started, if the capacity for it to run 
 all the time on a dedicated machine doesn't exist. 
 %this is formal english for: we do not have a server. Gibe server plos
The program needs a server to run. 

If a dedicated server exists, the program can be used from anywhere
 with a decent network connection to it.

If not, the program can still be run on the same computer as 
the server (on localhost), but the server will have to be started first.
 

