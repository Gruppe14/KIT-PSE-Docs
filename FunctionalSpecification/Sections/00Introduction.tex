\section*{Introduction}
PSE is a mandatory course for students of Bachelor Informatik in the 
Karlsruhe Institute of Technology (KIT). 
For this course we have to form groups of five or six people to write
 a program of 'medium size' ($\sim$5000 Lines of Code).

Our task given in this course consists of
analyzing and visualizing the server log of the 
\href{http://skyserver.sdss.org/public/en/}{SkyServer}.
This is one of the biggest public databases for astronomical data. 
It contains pictures and informations to many stuff you can see on the night sky
and tries to form a 'map of the universe'.

This assignment is one of only two which will be handled 
in English.
Because none of us speaks English as his first language, 
we want to apologize that our documentation may contain simple 
or maybe in some parts incorrectly used language. We focus more on 
having easy-to-understand and correct documentation 
than on perfect English, which sounds like written by a native speaker.

We are going to write a web-frontend in javascript allowing users 
to access the finished application from everywhere
with recent browsers. The actual application can be spliited into two 
different parts. The first part - which will be referred as 'Parser' and is going to convert the CSV-formatted 
serverlog from Skyserver into a data-warehouse. It will
be accessible via admin-login to the webpage where we can enter a 
logfile, which will be parsed into our warehouse.
The second part - which will be referred as 'Analyzer' works on this data-warehouse 
and allows every user of our webpage to create various charts, 
for example scatterplots and histograms.
Due to the fact that we split our project in two smaller parts, 
many parts of this specification are going to be splitted in two.

We have to state that no one of us has much experience in working 
with databases and javascript.
Therefore we can't guarantee for the correctness of all information 
stated in this specification.
It could happen that we will alter some of the specifications when designing or 
implementing the actual program.
