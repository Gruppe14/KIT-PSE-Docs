\section*{About this document}

<<<<<<< HEAD

\subsection*{General Information}
%\subsection*{Why this specification is written}
PSE is a mandatory course for students of Bachelor Informatik in the % Glossary: PSE, KIT
Karlsruhe Institute of Technology (KIT). Therefor groups of five to six
students are formed to write programs of 'medium size' (about 5000 Lines of Code).
=======
\subsection*{General Information}
PSE is a mandatory course for students of Bachelor Informatik in the % Glossary: PSE, KIT
Karlsruhe Institute of Technology (KIT). Therefor groups of five to six
students are formed to write programs of 'medium size' (about 5000 Lines of Code). 
>>>>>>> ce24c2b0e6cfab23d16ec248a9faa920f776c1ef
The task given in this course consists of analyzing and visualizing the server log of a database.

This specification is the one of PSE Group \#10 in the winter semester 2012/13.


<<<<<<< HEAD
%\subsection*{Content of this specification}
\subsection*{What this specification does}
The purpose of this document is a outline of the functional specifications and requirements
for the WHAT application. Is tries to give a complete and exact model of the future system.
In the course of this it wants to give the developers answers to all possible question concerning
what should be implemented.


\subsection*{What this specification does not}
This specification does not give any informations on how the system should be implemented. 
Also it doesn't contain any project planning or deadlines.
=======
\subsection*{Content of this specification}
The purpose of this document is a outline of the functional specifications and requirements
for the WHAT application. Is tries to give a complete and exact model of the future system.
In the course of this it wants to give the developers answers to all possible question concerning
the functions of WHAT.


\subsection*{What this specification does not}
This specification does not give any informations on how the system should be implemented. Also it doesn't 
contain any project planning or deadlines.
>>>>>>> ce24c2b0e6cfab23d16ec248a9faa920f776c1ef


\subsection*{A 'Living Document'}
Software development is a dynamic process. So with proceeding development the
content of this specification will change continuously. 


\subsection*{Skyserver}
The concept of the system described in this specification should work with any database
storing it's query-log. The \href{http://skyserver.sdss.org/public/en/}{SkyServer} will function
as example and testing reference for this system. Skyserver is one of the biggest databases for astronomical data.

\subsection*{About us}
%\subsection*{About the writters}
Because no one of us speaks English as his first language,
we can't guarantee that our documentation doesn't contain simple
or incorrectly used language. Our focus lies more on
having easy-to-understand and correct documentation
than on perfect English, which looks like written by a native speaker.

If you have any questions or comments regarding this document feel free to send us an 
 \href{mailto:pse10-group14-ws12@ira.uni-karlsruhe.de}{E-Mail}.

<<<<<<< HEAD
% \subsection*{Questions and comments}
% .
=======

\subsection*{About us}
Because no one of us speaks English as his first language, 
we can't guarantee that our documentation doesn't contain simple 
or incorrectly used language. Our focus lies more on 
having easy-to-understand and correct documentation 
than on perfect English, which looks like written by a native speaker.

 
\newpage
\section*{Introduction} % obsolet?
PSE is a mandatory course for students of Bachelor Informatik in the 
Karlsruhe Institute of Technology (KIT).
In this course we have to form groups of five or six people to write
 a program of 'medium size' ($\sim$5000 Lines of Code).

Our task given in this course consists of
analyzing and visualizing the server log of the 
\href{http://skyserver.sdss.org/public/en/}{SkyServer}.
This is one of the biggest public databases for astronomical data. 
It contains pictures and informations of astronomical data
and tries to form a 'map of the universe'.

This assignment is one of two assignments which will be handled 
in English.
Because no one of us speaks English as his first language, 
we want to apologize that our documentation may contain simple 
or in some parts incorrectly used language. Our focus lies more on 
having easy-to-understand and correct documentation 
than on perfect English, which looks like written by a native speaker.
>>>>>>> ce24c2b0e6cfab23d16ec248a9faa920f776c1ef




 
% \newpage
% \section*{Introduction} % obsolet?
% PSE is a mandatory course for students of Bachelor Informatik in the
% Karlsruhe Institute of Technology (KIT).
% In this course we have to form groups of five or six people to write
%  a program of 'medium size' ($\sim$5000 Lines of Code).
% 
% Our task given in this course consists of
% analyzing and visualizing the server log of the
% \href{http://skyserver.sdss.org/public/en/}{SkyServer}.
% This is one of the biggest public databases for astronomical data.
% It contains pictures and informations of astronomical data
% and tries to form a 'map of the universe'.
% 
% This assignment is one of two assignments which will be handled
% in English.
% Because no one of us speaks English as his first language,
% we want to apologize that our documentation may contain simple
% or in some parts incorrectly used language. Our focus lies more on
% having easy-to-understand and correct documentation
% than on perfect English, which looks like written by a native speaker.
% 
% We are going to write a web-frontend in javascript allowing users
% to access the finished application from everywhere
% with recent browsers. The actual application can be spliited into two
% different parts. The first part - which will be referred as 'Parser' and is going to convert the CSV-formatted
% serverlog from Skyserver into a data-warehouse. It will
% be accessible via admin-login to the webpage where we can enter a
% logfile, which will be parsed into our warehouse.
% The second part - which will be referred as 'Analyzer' works on this data-warehouse
% and allows every user of our webpage to create various charts,
% for example scatterplots and histograms.
% Due to the fact that we split our project in two smaller parts,
% many parts of this specification are going to be splitted in two.
% 
% We have to state that no one of us has much experience in working
% with databases and javascript.
% Therefore we can't guarantee the correctness of all information
% stated in this specification. Some of the specifications may be altered when we design or
% implement the actual program.
