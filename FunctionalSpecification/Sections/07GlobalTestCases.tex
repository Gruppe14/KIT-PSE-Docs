\section{Test Cases}


% Enumeration changed
% \renewcommand{\theenumi}{/T\arabic{enumi}0/}
% \renewcommand{\labelenumi}{\theenumi}

\subsection{General}

These test-cases are necessary for the webpage to work 
properly and succeed in serving basic requests.

\begin{enumerate}[label={{/T}\itemnumber*{/}}, series=test]

\item Access web page (\ref{f1})
\begin{enumerate}
  \item[/T012/] via Google Chrome,
  \item[/T014/] via Mozilla Firefox.
\end{enumerate}
\label{t10}

\item Navigate swiftly through the web page. (\ref{N1}) 
\label{t11}

\item Open page for scatter plot-creation and create an example
      with x-Axis date and y-Axis number of lines per request. (\ref{f2}, \ref{f3}, \ref{f7}, \ref{f9})
\label{t12}

\item Open page for histogram-creation and create an example
      with x-Axis year and y-Axis number of requests. (\ref{f2}, \ref{f3}, \ref{f4}, \ref{f6})
\label{t13}

\item Open page for bubble chart-creation and create an example
      with x-Axis year, y-Axis Country and size number of requests. (\ref{f2}, \ref{f3}, \ref{f10}, \ref{f11})
\label{t14}

\item Call the information about chart types. (\ref{f13})
\label{t15}

\item Request information about selectable variables. (\ref{f14})
\label{t16}

\end{enumerate}

\subsection{Administrator specific}

These test cases check the correct admin-login.

\begin{enumerate}[resume*=test]

\item Enter loginname and password for admin-login. (\ref{f15})
\label{t17}

\item Log-in and get administrative rights. (\ref{f15})
\label{t18}

\item Try to log in with wrong loginname. (\ref{f15})
\label{t19}

\item Try to log in with wrong password. (\ref{f15})
\label{t20}

\item Pass example log-file to the parser. (\ref{f16})
\label{t21}


\end{enumerate}

\newpage
\subsection{Parser specific}

The following test cases ensure the correct work of our parser.
%, which extracts information from a .csv-file and writes 
%it in our data-warehouse.

\begin{enumerate}[resume*=test]

\item Parse example log-file and extract valuable information correctly. (\ref{p1})
	\\ More specifically: 
\label{t22}
  \begin{enumerate}

	\item[/T132/] Extract correct database, time, user information 
	and type, (\ref{p2})
\label{t23}

	\item[/T134/] Extract correct number of rows, elapsed time, 
	busy time from the log-file, (\ref{f17})
\label{t24}	
	
	\item[/T136/] Extract the type requested. (\ref{p4})
\label{t25}

  \end{enumerate}

\item Load the extracted information into the data warehouse correctly. (\ref{f18}, \ref{f19})
\label{t26}

\end{enumerate}


\subsection{Analyzer specific}

The functions  \ref{f20} and \ref{f21} of the analyzer are 
already tested in the general test-cases \ref{t12} to \ref{t14}.



% \subsection{Scenario Testing}
% 
% Scenario testing is sequence Workflow from use cases and shod lead iteratively
% 
% % % Enumeration changed
% % \renewcommand{\theenumi}{/T\arabic{enumi}0/}
% % \renewcommand{\labelenumi}{\theenumi}
% 
% \subsubsection {Scenario 1: loading logs file} 
% 
% \begin{enumerate}
%  
% \item Start the Application
% 
% \item Log in the User with the Admin ID and password
% 
% \item Write the Queries
% 
% \item Send the Querise to Skyserver
% 
% \item Send the logs file to parse
% 
% \item Call the opportunity to initialize the data warehouse
% 
% \item Call the opportunity to request new log-files for specific time intervals from the
% database
% 
% \item Call the opportunity to clean the data warehouse %what's mean?
% 
% \item Leave the Application
% 
% \end{enumerate} 
% 
% \subsubsection {Scenario 2: parsing} 
% 
% \begin{enumerate}
% 
% \item Start the Application
% 
% \item Log in the User with the Admin ID and password
% 
% \item Call Extract specific data from the log-files
% 
% \item Call Extract access database, access time and user information (dimensions)
% 
% \item Call Extract number of rows, elapsed time, busy time (measures)
% 
% \item Call Extract type of data requested from the where-part
% 
% \item Call Transform the data to fit into the data warehouse schema
% 
% \item Load the data into the data warehouse
% 
% \item Leave the Application 
% 
% \end{enumerate} 
% 
% \subsubsection {Scenario 3: selecting variables of diagrams} 
% 
% \begin{enumerate}
% 
% \item Start the Application
% 
% \item Select a diagram type
% 
% \item Show the diagram
% 
% \item Call option for the tow or three variables of the diagram
% 
% \item Change one, tow or three variables of the diagram
% 
% \item Refresh the Information
% 
% \item Show the new diagram with new variables of the diagram
% 
% \item Leave the Application
% 
% \end{enumerate}
% 
% \subsubsection {Scenario 4: options} 
% 
% \begin{enumerate}
% 
% \item Start the Application
% 
% \item Call the option menu
% 
% \item Select the Languge of the Application
% 
% \item Show the Application with the selected languge
% 
% \item Leave the Application
% 
% \item Start the Application 
% 
% \item Select the diagram 
% 
% \item Call file menu
% 
% \item Save the diagram as a PDF file
% 
% \item Leave the Application
% 
% \end{enumerate}
% 
% \subsubsection {Scenario 5: A!} 
% 
% \begin{enumerate}
% 
% \item blabla
% 
% \end{enumerate}
