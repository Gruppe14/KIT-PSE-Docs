\section{Test Cases}

\subsection{Global test cases}

is a set of conditions and variables under which a test cases will determine whether an this Application is working correctly 
or not.

% Enumeration changed
\renewcommand{\theenumi}{/T\arabic{enumi}0/}
\renewcommand{\labelenumi}{\theenumi}

\subsubsection{General}

Those test-cases are necessary for the webpage to work properly and succeed in serving basic requests.

\begin{enumerate}

\item Access webpage with Mozilla Firefox 16 \& 
Google Chrome 23 (local or internet) (\ref{f1})

\item Open chart-type menu (unnecessary if chart-type menu is on the first page) (\ref{f2})

\item Open window for scatterplot-creation (\ref{f2}) and create an example (\ref{f7})
      with x-Axis date and y-Axis number of lines per request (\ref{f8}) (\ref{f9})

\item Open window for histogram-creation(\ref{f2}) and create an example (\ref{f4})
      with x-Axis year and y-Axis number of requests (\ref{f5}) (\ref{f6})

\item Open window for bubble-chart-creation(\ref{f2}) and create an example (\ref{f10})
      with x-Axis year, y-Axis Country and size number of requests (\ref{f11}) (\ref{f12})

\item Call the information about chart types (\ref{f13})

\item Request information about selectable variables (\ref{f14})

\end{enumerate}

\subsubsection{Administrator specific}

Those test cases check the correct admin-login.

\begin{enumerate}

\item Enter loginname and password for admin-login. (\ref{f15})

\item Log-in and get administrative rights. (\ref{f15})

\item Try to log in with wrong loginname (\ref{f15})

\item Try to log in with wrong password (\ref{f15})

\item Pass example log-file to the parser (\ref{f16})

\end{enumerate}

\subsubsection{Parser specific}

The following test-cases test the correct work of our parser, which extracts information from a .csv-file and writes 
it in our data-warehouse.

\begin{enumerate}

\item Parse example log-file and extract valuable information correctly. (\ref{f17}) More specifically :

\begin{itemize}

\item Extract correct database, time, user information and type of request from the log-file (\ref{f18}) (\ref{f20})

\item Extract correct number of rows, elapsed time, busy time from the log-file(\ref{f18})

\end{itemize}

\item Write extracted information in the warehouse correctly. (\ref{f21}) (\ref{f22})

\end{enumerate}


\subsubsection{Analyzer specific}

The analyzer is already tested in the general test-cases.



\subsection{Scenario Testing}

Scenario testing is sequence Workflow from use cases and shod lead iteratively

% Enumeration changed
\renewcommand{\theenumi}{/T\arabic{enumi}0/}
\renewcommand{\labelenumi}{\theenumi}

\subsubsection {Scenario : load logs file} 

\begin{enumerate}
 
\item Start the Application

\item Log in the User with the Admin ID and password

\item Write the Queries

\item Send the Querise to Skyserver

\item Send the logs file to parse

\item Call the opportunity to initialize the data warehouse

\item Call the opportunity to request new log-files for specific time intervals from the
database

\item Call the opportunity to clean the data warehouse %what's mean?

\item Leave the Application

\end{enumerate} 

\subsubsection {Scenario : parser} 

\begin{enumerate}

\item Start the Application

\item Log in the User with the Admin ID and password

\item Call Extract specific data from the log-files

\item Call Extract access database, access time and user information (dimensions)

\item Call Extract number of rows, elapsed time, busy time (measures)

\item Call Extract type of data requested from the where-part

\item Call Transform the data to fit into the data warehouse schema

\item Load the data into the data warehouse

\item Leave the Application 

\end{enumerate} 

\subsubsection {Scenario : select the variables of diagram} 

\begin{enumerate}

\item Start the Application

\item Select a diagram type

\item Show the diagram

\item Call option for the tow or three variables of the diagram

\item Change one, tow or three variables of the diagram

\item Refresh the Information

\item Show the new diagram with new variables of the diagram

\item Leave the Application

\end{enumerate}

\subsubsection {Scenario : Option} 

\begin{enumerate}

\item Start the Application

\item Call the option menu

\item Select the Languge of the Application

\item Show the Application with the selected languge

\item Leave the Application

\item Start the Application 

\item Select the diagram 

\item Call file menu

\item Save the diagram as a PDF file

\item Leave the Application

\end{enumerate}

\subsubsection {Scenario : A} 

\begin{enumerate}

\item blabla

\end{enumerate}
