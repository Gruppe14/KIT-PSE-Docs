\section{Global test cases}

is a set of conditions and variables under which a test cases will determine whether an this Application is working correctly 
or not.

% Enumeration changed
\renewcommand{\theenumi}{/T\arabic{enumi}0/}
\renewcommand{\labelenumi}{\theenumi}

\subsection{Test case}

is the description of Workflow of this Application and depend on the Functional requirements. This Workflow must work
successful and correctly with alle these points. 

\begin{enumerate}


\item Start internet browser (\ref{f1})

\item Type the web address - say pse10-analyzer.edu - in browser's address bar (\ref{f1})

\item Call the Home Page with a Table Bottun /F20/ % What does this mean?

\item Select a Type of chart /F30/

\item Call the diagram Page 
% may get obsolot if we start with the overview as planed

\item Call histogram /F40/

\item Call the option for tow variables of the histrogram /F50/

\item Call the option for the interval or selection of the x-axis variable /F60/

\item Call  scattered plots  /F70/

\item Call the option for tow variables of the histrogram /F80/

\item Call the option for the interval or selection of the x-axis variable /F90/

\item Call  scattered plots  /F70/

\item Call the option for tow variables of the scattered plots /F80/

\item Call the option for the interval or selection of tow variables /F90/

\item Call bubble charts /F100/

\item Call the option for three variables of the scattered plots /F110/

\item Call the option for the interval or selection of tow variables /F120/

\item Call the information about chart types (\ref{f13})

\item Call information about selectable variables (\ref{f14})

\item Leave the web page

\end{enumerate}




\subsection{Scenario Testing}

Scenario testing is sequence Workflow from use cases and shod lead iteratively

% Enumeration changed
\renewcommand{\theenumi}{/T\arabic{enumi}0/}
\renewcommand{\labelenumi}{\theenumi}

\subsubsection {Scenario : load logs file} 

\begin{enumerate}

\item Start the Application

\item Log in the User with the Admin ID and password

\item Write the Queries

\item Send the Querise to Skyserver

\item Send the logs file to parse

\item Call the opportunity to initialize the data warehouse

\item Call the opportunity to request new log-files for specific time intervals from the
database

\item Call the opportunity to clean the data warehouse %what's mean?

\item Leave the Application

\end{enumerate}

\subsubsection {Scenario : parser} 

\begin{enumerate}

\item Start the Application

\item Log in the User with the Admin ID and password

\item Call Extract specific data from the log-files

\item Call Extract access database, access time and user information (dimensions)

\item Call Extract number of rows, elapsed time, busy time (measures)

\item Call Extract type of data requested from the where-part

\item Call Transform the data to fit into the data warehouse schema

\item Load the data into the data warehouse

\item Leave the Application

\end{enumerate}

\subsubsection {Scenario : select the variables of diagram} 

\begin{enumerate}

\item Start the Application

\item Select a diagram type

\item Show the diagram

\item Call option for the tow or three variables of the diagram

\item Change one, tow or three variables of the diagram

\item Refresh the Information

\item Show the new diagram with new variables of the diagram

\item Leave the Application

\end{enumerate}

\subsubsection {Scenario : Option} 

\begin{enumerate}

\item Start the Application

\item Call the option menu

\item Select the Languge of the Application

\item Show the Application with the selected languge

\item Leave the Application

\item Start the Application 

\item Select the diagram 

\item Call file menu

\item Save the diagram as a PDF file

\item Leave the Application

\end{enumerate}

\subsubsection {Scenario : A} 

\begin{enumerate}

\item blabla

\end{enumerate}
